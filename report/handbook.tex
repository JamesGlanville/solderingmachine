% File name: report/handbook.tex

\documentclass[a4paper,11pt]{article}  % Standard document class
\usepackage[english]{babel}            % Set document language
\usepackage{fullpage}                  % Set up page for small margins etc
\usepackage{listings}
\usepackage{graphicx}                  % For including images in document

\usepackage{tikz}
\usetikzlibrary{shapes,arrows}

\usepackage{appendix}
\usepackage{color}
\usepackage{hyperref}

\usepackage{caption}
\usepackage{subcaption}

%\usepackage{placeins}                  % Allows use of \FloatBarrier
% to avoid images or tables
% moving into next section
%\usepackage{subfig}                    % For subfigures...

\usepackage{amsmath}                   % For improving maths/formula typesetting
%\usepackage{tabular}                  % Table changing package

%\usepackage{algpseudocode}             % For producing algorithms/flowcharts

\usepackage{listings}                  % For including source code in document
\lstset{
  basicstyle = \small
}

% Provide command for scientific notation
\providecommand{\e}[1]{\ensuremath{\times10^{#1}}}
\providecommand{\degrees}{\ensuremath{^{\circ}}}

% Define title here:
\title{Soldering Machine Handbook}
\author{James Glanville}
\date{15th May 2013}

\hypersetup{
    colorlinks,
    allcolors=blue,
    linktoc=all,
}

\begin{document}

% generate title
\maketitle

\tableofcontents

\newpage

\section{Introduction}

This handbook lists the steps required to operate the machine, and
produce populated PCBs. The operating system used in this handbook is
Ubuntu 13.10. The PCB CAD program used is Kicad.

\section{Prerequisites}
The following things are assumed:

\begin{itemize}
	\item	Computer with Windows/Linux. OS X may work, but is untested.
	\item	SD card (at least 128mb).
\end{itemize}

\section{Installing Software}
The following steps need only be carried out once.

\subsection{Installing prerequisites}
The following command will install the necessary packages:

\begin{lstlisting}
sudo apt-get install git build-essential
\end{lstlisting}

\subsection{Obtaining software}
The following commands will fetch the necessary software from source control.

\begin{lstlisting}
git clone https://github.com/JamesGlanville/pcb2gcode-metric.git
git clone https://github.com/JamesGlanville/solderingmachine.git
\end{lstlisting}

\subsection{Compiling Software}
The following command will compile pcb2gcode:
\begin{lstlisting}
cd pcb2gcode-metric
./git-build.sh
\end{lstlisting}

\subsection{Install pronterface (if PC control is required)}
\begin{lstlisting}
git clone https://github.com/kliment/Printrun.git
\end{lstlisting}


\subsection{Preparing SD card}
Format the SD card to the FAT32 filesystem, ensuring the disk has an
MSDOS partition table and that the FAT32 partition is on the first
primary partition.

In Ubuntu, the following steps are necessary:

Install gparted if it is not already installed:

\begin{lstlisting}
sudo apt-get install gparted
\end{lstlisting}

Launch gparted as root:
\begin{lstlisting}
sudo gparted
\end{lstlisting}

Select the SD card using the drop down menu at the top right of the
screen.

Create the MSDOS partition table by clicking Device-\textgreater Create Partition Table.

Create the FAT32 filesystem by right-clicking on the empty disk, and selecting "New". 
Fill in the information in the dialogue box, and click "Apply changes". 

\subsection{Copying utility GCODE to SD card.}
Copy all .GCODE files from pcb2gcode-metric/GCODE/ to the root of the SD card.
These can be used to perform simple functions such as zeroing the axes.

\section{Creating GERBER files.}
The following steps should be followed to export GERBER files from Kicad:

\begin{enumerate}
	\item	Open the PCB layout in PCBNew.
	\item	Select \textbf{Tracks and Vias} from the \textbf{Dimensions} Menu.
	\item	Change the \textbf{Mask Clearance} to 0.0001.
	\item	Select \textbf{Plot} from the \textbf{File} menu.
	\item	Select the following layers: \textbf{Copper, Component, SoldP_Cmp, Edges_Pcb}
	\item	Click \textbf{Plot} and then \textbf{Generate drill file}.
\end{enumerate}

\section{Generating GCODE with pcb2gcode}
\subsection{Creating millproject file.}
An example millproject file is given in pcb2gcode-metric/millproject. This should be copied
into the directory containing the GERBER files, and modified if necessary (see included comments).

\subsection{Copy GCODE to SD Card}

\section{Isolation Routing}
Attach copper-clad PCB to machine (how?).

\subsection{Configuring the machine}
Insert conical routing bit into the spindle. This is achieved by loosening
the idler section, and then loosening the two machine screws holding the bit in place.
The bit should be carefully removed by hand, and the conical routing bit 
pushed into its place. The two machine screws should be tightened, and the idler
tightened.

The spindle tool should then be attached to the tool holder with the two
wing nuts.

\subsection{Routing}
Run "zero.gcode" then "startspindle.gcode" then "mill.gcode"

Run "stopspindle.gcode"

\section{Drilling}
This section may be ignored if no holes need be drilled through the PCB.

\subsection{Configuring the machine}
The drill bit should be inserted into the spindle tool as described in 
the "Isolation Routing" section above.

\subsection{Drilling}
Run "zero.gcode" then "startspindle.gcode" then "drill.gcode"

Run "stopspindle.gcode"

\section{Paste Extrusion}

\subsection{Configuring the machine}
Insert a syringe containing a sufficient volume of solder paste into
the paste extruder. Hand-tighten the extruder so that there is only a 
small amount of slack in the timing belt.

Attach the paste extruder to the tool holder and hand-tighten the wing 
nuts.




\newpage
\appendix
\appendixpage
\addappheadtotoc






\section{NEED TO DO}
I need to describe how zeroing works, with attaching electrical clips.

\newpage
\section{Building pcb2gcode from source}

The pcb2gcode source should first be obtained:
\begin{lstlisting}[frame=single]
git clone git@github.com:JamesGlanville/pcb2gcode-metric.git
\end{lstlisting}

\subsection{Compilation in Linux}
For these instructions the use of Ubuntu/Debian is assumed, although other distributions should be similar.

To install prerequisites:
\begin{lstlisting}[frame=single,breaklines=true]
sudo apt-get install build-essential automake autoconf libtool libboost-all-dev libgtkmm-2.4-dev gerbv
\end{lstlisting}

To compile pcb2gcode:
\begin{lstlisting}[frame=single]
cd pcb2gcode-metric
./git-build.sh
\end{lstlisting}

(Optional) To install pcb2gcode:
\begin{lstlisting}[frame=single]
sudo make install
\end{lstlisting}

\subsection{Compilation in Windows}
Installing prerequisites is more complex in Windows:

\subsubsection{Visual Studio}
Install Visual Studio 2012 Professional from:
\begin{lstlisting}[frame=single]
www.visualstudio.com
\end{lstlisting}

\subsubsection{GTK+ development files}
Download:
\begin{lstlisting}[frame=single,breaklines=true]
http://ftp.gnome.org/pub/gnome/binaries/win32/gtk+/2.24/gtk+-bundle_2.24.10-20120208_win32.zip
\end{lstlisting}
and extract to pcb2gcode-metric folder.

\subsubsection{Gerbv}
download gerbv from http://sourceforge.net/projects/gerbv/
gerbvinst-2.6.1.exe and install
TODO: add instructions for generating .lib file

\subsubsection{Boost libraries}
\begin{lstlisting}[frame=single,breaklines=true]
download boost_1_55_0.zip from http://sourceforge.net/projects/boost/
\end{lstlisting}
unzip contents to c:\\boost155
open VS2012 x86 Native Tools Command Prompt, from C:\\ProgramData\\Microsoft\\Windows\\Start Menu\\Programs\\Microsoft Visual Studio 2012\\Visual Studio Tools
AS ADMINISTRATOR
cd c:\\boost155
bootstrap.bat
bjam.exe

\subsubsection{GTKMM}
\begin{lstlisting}[frame=single,breaklines=true]
download http://ftp.gnome.org/pub/GNOME/binaries/win32/gtkmm/2.22/gtkmm-win32-devel-2.22.0-2.exe
\end{lstlisting}
from http://ftp.gnome.org/pub/GNOME/binaries/win32/gtkmm/2.22/
and install.

\subsubsection{gerbv?}
\begin{lstlisting}[frame=single,breaklines=true]
download gerbv-2.6.1.tar.gz from http://sourceforge.net/projects/gerbv/files/gerbv/gerbv-2.6.1/
\end{lstlisting}
and extract to c:\\gerbv-2.6.1

hack appwindow.c
create config.h
create demo-config.h
%\end{verbatim}

\end{document}
