% File name: report/repord.tex

\documentclass[a4paper,11pt]{article}  % Standard document class
\usepackage[english]{babel}            % Set document language
\usepackage{fullpage}                  % Set up page for small margins etc

\usepackage{graphicx}                  % For including images in document
%\usepackage{placeins}                  % Allows use of \FloatBarrier
% to avoid images or tables
% moving into next section
%\usepackage{subfig}                    % For subfigures...

\usepackage{amsmath}                   % For improving maths/formula typesetting
%\usepackage{tabularx}                  % Table changing package

%\usepackage{algpseudocode}             % For producing algorithms/flowcharts

\usepackage{listings}                  % For including source code in document
\lstset{
  basicstyle = \small
}

% Provide command for scientific notation
\providecommand{\e}[1]{\ensuremath{\times10^{#1}}}
\providecommand{\degrees}{\ensuremath{^{\circ}}}

% Define title here:
\title{4th Year Project: Soldering Machine}
\author{James Glanville}
\date{11th June 2013}

\begin{document}

% generate title
\maketitle

\section{Existing Solutions}
There are a number of hobbyist solutions to deal with SMD parts:

\begin{itemize}
	\item	Solder paste + heat:
		\begin{itemize}
			\item	Solder stencils: Expensive setup costs, very quick to use. Not suitable for
				this project.
			\item	Manual solder paste application: Mostly expensive dispenser, or with a syringe.
			\item	Hot air gun: Can be relatively difficult to get the right temperature profile.
			\item	Converted toaster over: seems quite common.
		\end{itemize}

	\item	Soldering by hand:
		\begin{itemize}
			\item	Dragging a bead of solder along pins, then cleaning up with solder sucker/wick.
				Too much feedback required to automate?
			\item	Some devices (SOIC?) have slightly bendy pins. A small amount of downwards 
				pressure on the pin onto a tinned pad works.
		\end{itemize}
\end{itemize}

\section{Design}

\subsection{X/Y axes}

The X/Y axes have the same requirements. ~0.5mm pitch smd devices common - 0.05mm repeatability reasonable
target? Choice between stationary "bed" and moving tools, or vice-versa. Stationary bed reduces likelihood
of jolting parts around, but tools are bulkier and heavier than the bed, which increases demands on driving
mechanism. I quite like the idea 

Driving mechanisms:

\begin{itemize}
	\item	Toothed belts + pulleys + stepper motors: Simple, as used in repraps. Can be run open loop
		very easily. Requires: stepper motor + driver, belt, pulley.
	\item	Threaded rod + stepper motors: Cheap, slower but probably fast enough. M3-5 easy to couple
		to motor shafts (http://www.thingiverse.com/thing:9622).
	\item	Closed loop: dc motors + feedback. Linear potentiometers: relatively expensive, and potential
		issues with electrical noise. Rotary encoders: cheap, accurate (m4 pitch is ~0.5mm, not much
		travel/turn, might only need 10 slot encoder)

\end{itemize}

Linear slides:

\begin{itemize}
	\item	Drawer slides: cheap, surprisingly high precision.
	\item	3d printed bushings + steel rod. If 3d printer existance assumed, then very cheap solution.
	\item	LM8UUs (or smaller): About £1 each, so probably too expensive.
\end{itemize}

\subsection{Z axis}
The Z axis will not require as much precision as the X and Y. Potential mechanisms:

\begin{itemize}
	\item	Micro servos: cheap (£2 on ebay). Require no drivers, and simple to drive. Z axis potentially
		does not need to be linear (UP/DOWN only?) so rotary->linear mechanism simpler. If not linear,
		then hinging out of the way is a cheaper mechanism than slides.
\end{itemize}

\subsection{Part placement}

\begin{itemize}
	\item	Manual placement. Much more of a problem for a large number of SMD resistors/capacitors etc, than
		larger components. 
	\item	Vacuum "tweezers". Mechanically relatively simple, but problems of intelligent control, as well
		as well as storing the components before placement. As an experiment, I tried using an aquarium pump
		in reverse (by attaching the silicone tubing to the inlet instead of the outlet) connected to a fairly
		large diameter needle (ID: Xmm).
\end{itemize}

\subsection{Heated Bed}
Soldering to a hot board is easier (smaller temperature difference). Probably not worth the cost/complexity.

\subsection{Flux application}
Flux application may be a significant problem, unless a manual brush with a solder pen is sufficient.

\subsection{Soldering iron}
Designing a mount for a cheap and widely available soldering iron may well be cheapest. 

\subsection{Electronics/Firmware}


\section{Tools}
I have a small mill, a 3d printer and a lathe. It may be worth considering that a lot of people/schools have
access to a laser cutter and perhaps mill/lathe, so if possible all custom components could be 3d printed or 
lasercut.


\end{document}
